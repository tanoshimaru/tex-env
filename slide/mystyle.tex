\ProvidesPackage{mystyle}

% for math
\usepackage{amsmath}
\usepackage{amssymb}
\usepackage{bm}
\usepackage{ascmac}
\usepackage{mathtools}
\usepackage{amsfonts}

% for table
\usepackage{booktabs}
\usepackage{multicol}
\usepackage{multirow}

% for figures
\usepackage{graphicx}

% for text color
\usepackage{xcolor}
\newcommand{\red}[1]{\textcolor{red}{#1}}

% for bibliography
\usepackage[backend=biber, style=numeric, sortcites=true, sorting=nyt]{biblatex}
\setbeamertemplate{bibliography item}[text]
\addbibresource{../paper/ref.bib}

% for hyperref
\usepackage{hyperref}

% for algorithm
\usepackage[noend]{algorithmic}
\usepackage{algorithm}
\renewcommand{\algorithmicrequire}{\textbf{Input:}}
\renewcommand{\algorithmicensure}{\textbf{Output:}}

% for cryptocode
\usepackage[keys]{cryptocode}
\createpseudocodeblock{pcb}{center, boxed}{}{}{linenumbering}

% % for theorem
% \theoremstyle{definition}
% \newtheorem{theorem}{定理}
% \newtheorem{prop}[theorem]{命題}
% \newtheorem{lemma}[theorem]{補題}
% \newtheorem{cor}[theorem]{系}
% \newtheorem{example}[theorem]{例}
% \newtheorem{definition}[theorem]{定義}
% \newtheorem{remark}[theorem]{注意}
% \newtheorem{guide}[theorem]{参考}

% for referencing
\newcommand\figref[1]{\textbf{図~\ref{fig:#1}}}
\newcommand\tabref[1]{\textbf{表~\ref{tab:#1}}}
\newcommand\algref[1]{\textbf{アルゴリズム~\ref{alg:#1}}}
\newcommand\chapref[1]{~\ref{chap:#1}章}
\newcommand\secref[1]{~\ref{sec:#1}節}

% for spacing
\usepackage{setspace}

% for fonts
\usepackage{xeCJK}
\usepackage{fontspec}
% \setmainfont{NotoSans-Regular}  % コンパイルに時間がかかるので一旦コメントアウト,最後にコメントアウトを外す
\setCJKmainfont{Noto Sans CJK JP}
\setCJKsansfont{Noto Sans CJK JP}

\usetheme{metropolis}
\metroset{
  block=fill, % ブロックに背景をつける
  numbering=fraction % 合計ページ数を表示
}
% \setbeamerfont{frame numbering}{size=\large} % ページ番号のフォントサイズ
\usefonttheme{professionalfonts} % 数式のフォントを変更

% % テーマカラーを設定
% \definecolor{kuro}{HTML}{282c34}
% \setbeamercolor{structure}{fg=kuro}
% \setbeamercolor{section in toc}{fg=kuro}
% \setbeamercolor{subsection in toc}{fg=kuro}
% \setbeamercolor{frametitle}{bg=kuro}
% \setbeamercolor{title}{fg=kuro}
% \setbeamercolor{author}{fg=kuro}
% \setbeamercolor{date}{fg=kuro}
% \setbeamercolor{institute}{fg=kuro}
% \setbeamercolor{block title}{fg=kuro}
% \setbeamercolor{block body}{fg=kuro}
% \setbeamercolor{item}{fg=kuro}
% \setbeamercolor{itemize item}{fg=kuro}
% \setbeamercolor{itemize subitem}{fg=kuro}
% \setbeamercolor{itemize subsubitem}{fg=kuro}
% \setbeamercolor{enumerate item}{fg=kuro}
% \setbeamercolor{enumerate subitem}{fg=kuro}
% \setbeamercolor{enumerate subsubitem}{fg=kuro}
% \setbeamercolor{caption}{fg=kuro}
% \setbeamercolor{caption name}{fg=kuro}
% \setbeamercolor{caption title}{fg=kuro}
% \setbeamercolor{caption text}{fg=kuro}
% \setbeamercolor{caption separator}{fg=kuro}
% \setbeamercolor{caption number}{fg=kuro}

% itemize のトグルを変更
\setbeamertemplate{itemize item}{$\bullet$}
\setbeamertemplate{itemize subitem}{$\circ$}
\setbeamertemplate{itemize subsubitem}{--}

% tableofcontents のトグルを変更
\setbeamertemplate{section in toc}{\inserttocsectionnumber. \inserttocsection}
\setbeamertemplate{subsection in toc}{\hspace{1.2em}\inserttocsubsectionnumber. \inserttocsubsection}
\setbeamertemplate{subsubsection in toc}{\hspace{2.4em}\inserttocsubsubsectionnumber. \inserttocsubsubsection}

\AtBeginSection[]{
  \begin{frame}{構成}
    \tableofcontents[currentsection]
  \end{frame}
}

\endinput
